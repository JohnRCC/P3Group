\title{}
\author{}


%  This section deal with the program to solve electrostatic systems in C++ for the P3 Theoretical Laboratory 2015.
%
%	The text below decribes the main procedure as was the state of the program on:	12 - Feb - 2015
%	
%	 All information that is capitalized and within brackets is to be discussed and ammended depending on the
%	  development of the script. That is, they stand for parts that might be implemented in due course.
%


\documentclass{article}

\begin{document}


  \noindent\textbf{This is for obtaining the electric potential in a generic two-dimensional electrostatics system}

  \vspace{5mm}

	A bitmap (pixelated drawing) describing the components of the system is passed to the program. Each pixel is then 
	 mapped to a matrix holding values between 0 and 1, to represent the initial system. Values related to components
	 (conductors or charges) are static and the rest of space will vary as the prgram runs.\\
	A numerical algorithm, alg.() is run, "propagating" the field in the matrix, until the change in all values, is 
	 considered sufficiently small. The small change is determined by a user-defined percentage value. \\
	The matrix had thus a second layer; upon each iteration, the new values are cast to the other layer of the 
	 matrix and recast in the next instance of the application of the algorithm. 
	-1- When the system has settled, regions where discontinuities might be present, by, for example,
	 steep gradients, are treated, pixelwise, in a finer grid usng the same algorithm as above. -- OR: \\
	-2- When the system is considered stable, the elements in the matrix are further granulated depending on the 
	 steepness of their gradients. --. \\
	After the scheme is finished, the resulting matrix is then plotted using a plotting package where  a "heat-map" 
	 functionality is used to depict the potential. For the fieldlines, some further calculation is requred: the 
	 gradient between each pixel is calculated and used to plot the electric field, as requred by equation Eq.().

\vspace{7.5mm}

\noindent\textbf{The same as above separated into sections}

\vspace{5mm}

  \begin{tabular}{ l p{10cm} }
	Input		&	A bitmap (pixelated drawing) describing the components of he system is passed to the program. 
				 Each pixel is then mapped to a matrix holding values between 0 and 1, to represent the 
				  initial system. 
				 Values related to components (conductors or charges) are static and the rest of space will 
				  vary as the prgram runs.\\
	Algorithm	&	A numerical algorithm, alg.() is run, "propagating" the field in the matrix, until 
				  the change in all values, is considered sufficiently small. 
				 The small change is determined by a user-defined percentage value. 
				 The matrix had thus a second layer; upon each iteration, the new values are cast
				  to the other layer of the matrix and recast in the next instance of the application 
				  of the algorithm.\\
 	Meshing		&	-1- When the system has settled, regions where discontinuities might be present, by, for example,
	 			  steep gradients, are treated, pixelwise, in a finer grid usng the same algorithm as above. -- 
	 			  OR: 
				-2- When the system is considered stable, the elements in the matrix are further granulated 
				 depending on the steepness of their gradients. --. \\
	Results		&	After the scheme is finished, the resulting matrix is then plotted using a plotting 
				  package where a "heat-map" functionality is used to depict the potential. 
				 For the fieldlines, some further calculation is requred: the gradient between 
				  each pixel is calculated and used to plot the electric field, as requred by equation 
				  Eq.().
  \end{tabular}


\end{document}
