\title{}
\author{}


\documentclass{article}

\begin{document}


  \noindent\textbf{This is for obtaining the electric potential in a generic two-dimensional electrostatics system}

  \vspace{5mm}

	A bitmap (pixelated drawing) describing the components of he system is passed to the program. Each pixel is then 
	 represented by a value, between 0 and 1, in a matrix based on the map. Values related to components (conductors 
	 or charges) are static and the rest of space will vary as the prgram runs. \\
	A numerical algorithm is run, "propagating" the field in the matrix, until the change in all values, is considered sufficiently
	 small. This is defined by the requried uncertainty, $\epsilon$. \\
	After the scheme is finished, the resulting matrix is then plotted using a plotting package where  a "heat-map" functionality
	 is used to depict the potential. 

\vspace{7.5mm}

\noindent\textbf{The same as above separated into sections}

\vspace{5mm}

  \begin{tabular}{ l p{10cm} }
	Input		&	A bitmap (pixelated drawing) describing the components of he system is passed to the program.
				 Each pixel is then represented by a value, between 0 and 1, in a matrix based on the map. 
				 Values related to components (conductors or charges) are static and the rest of space will vary
				  as the prgram runs. \\
	Algorithm	&	A numerical algorithm is run, "propagating" the field in the matrix, until the change in all values, 
 				  is considered sufficiently small. This is defined by the requried uncertainty, $\epsilon$. \\
	Results	&	After the scheme is finished, the resulting matrix is then plotted using a plotting package where
				  a "heat-map" functionality is used to depict the potential. 
  \end{tabular}


\end{document}
