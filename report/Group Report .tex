\documentclass[a4paper]{jpconf}
\usepackage{amsmath} % for subequations
\usepackage{xfrac} %to write fractions in different ways
%\usepackage{graphicx} %for figures
%\usepackage{caption} %for figures



\begin{document}

\title{The effect of solid objects on the electrostatic potential}
\author{J Cork, T  Hendrick-Beattie, D Lafferty, M Pereira, V Nordgren and N Warrack}
\address{School of Physics and Astronomy, University of Glasgow, Glasgow, UK}

\begin{abstract}
\end{abstract}

\section*{Introduction}
Electromagnetism is one of four fundamental forces of nature. It describes how electrically charged particles interact with each other and how they generate electromagnetic fields \cite{Sears.Zamansky-uniPhy}. These fields permeate the space around them, influencing the behaviour of other charges by exerting forces on them. %(that is either attractive or repulsive). 
Electromagnetic forces determine the atomic and macroscopic properties of matter that dominate most of the physical phenomena encountered in daily life such as sound, light and biological processes. Understanding these principles in depth can facilitate significant advances in science and technology. For instance, successful modelling of a simple configuration of conductors makes it possible to investigate arbitrarily complex systems of charges aiding the design of detectors for particle physics. 


%EXTRA
%The electromagnetic force plays a major role in determining the internal properties of most objects encountered in daily life. Ordinary matter takes its form as a result of intermolecular forces between individual molecules in matter. Electrons are bound by electromagnetic wave mechanics into orbitals around atomic nuclei to form atoms, which are the building blocks of molecules. This governs the processes involved in chemistry, which arise from interactions between the electrons of neighboring atoms, which are in turn determined by the interaction between electromagnetic force and the momentum of the electrons.

Electrostatics is the study of stationary or slow-moving electric charges \cite{griffiths-introElec}. They exert electrostatic forces on each other, which in turn are governed by Coulomb's law and most conveniently described by electric field equations. The relationship between these fields and the distribution of electric charge can be expressed as \cite{griffiths-introElec}
\begin{equation}
\textbf{$\nabla$} \cdot \textbf{E} = \frac{\rho}{\epsilon_0}~,
%\oint \textbf{E} \cdot d\textbf{A} = \frac{Q_{enc}}{\epsilon_0}
\label{eq:intro1}
\end{equation} known as the differential form of Guass's law, where \textbf{$\nabla$} $\cdot$ \textbf{E} is the divergence of the electric field, $\epsilon_0$ is the permittivity of free space and $\rho$ is the charge density. Additionally, for any static charge distribution the electric field is irrotational, \textbf{$\nabla$} $\times$ \textbf{E} $= 0$. Therefore, the line integral of the electric field, $\oint \textbf{E} \cdot d \textbf{l} = 0$, is independent of the path taken \cite{griffiths-introElec}. This means that \textbf{E} can be written as the gradient of a scalar potential 
\begin{equation}
\textbf{E} = - \nabla \Phi~,
\label{eq:intro2}
\end{equation} where $\Phi$ is called the electric potential, defined as the potential energy per unit charge  \cite{Sears.Zamansky-uniPhy}.
In regions where there is no charge, such that $\rho = 0$, the divergence of the electric field is zero. Hence, using Eq.(\ref{eq:intro1}) and Eq.(\ref{eq:intro2}) the electric potential for any point in space can be described by
\begin{equation}
\nabla^2 \Phi = 0~.
\label{eq:intro3}
\end{equation} This is Laplace's equation \cite{RHB-MathematicalMethods}.\\ \par
In this work, the electrostatic potential and the electric field are evaluated at all points in two different systems. The systems consist of different geometric configurations of conductors held at constant potentials. This study shows how the potentials are modified in the presence of these conductors. 

\section*{Methods}
In general, there are two methods to analyse the electric potential in all space: analytical and numerical techniques. Often there is no analytical solution, even for a simple electrostatic system, and thus numerical methods are useful tools in obtaining approximate solutions. 

\subsection*{Analytical Approach} 
%Partial differential equations (PDE) are fundamental to describe physical phenomena, such as heat, fluids, electrostatics, sound and so on. This is an equation that involves an unknown function of two or more variables and their partial derivatives. A particular solution may be specified from the general solution of a PDE in the presence of boundary conditions. \par

The electric field is dominated by inherent symmetries that are better described in spherical coordinates \cite{RHB-MathematicalMethods} in which Laplace's Equation is expressed as

\begin{equation}
\frac{1}{r^2}\frac{\partial}{\partial r}\bigg(r^2 \frac{\partial V}{\partial r}\bigg) + \frac{1}{r^2 \sin \theta} \frac{\partial}{\partial \theta}\bigg(\sin \theta \frac{\partial V}{\partial \theta}\bigg) + \frac{1}{r^2 \sin^2 \theta}\frac{\partial^2 V}{\partial \phi^2} = 0~.
\label{eq:1}
\end{equation}

For round objects $V$ can be independent of $\phi$ given an appropriate choice of coordinate orientation. Multiplying through by $r^2$ the equation is simplified to
\begin{equation}
\frac{\partial}{\partial r}\bigg(r^2 \frac{\partial V}{\partial r}\bigg) + \frac{1}{\sin \theta}\frac{\partial}{\partial \theta}\bigg(\sin \theta \frac{\partial V}{\partial \theta}\bigg) = 0~.
\label{eq:2}
\end{equation}

Separation of variables is used to solve the equation above. Assume that $V(r,\theta) = R(r)\Theta(\theta)$, where the factors $R$ and $\Theta$ are functions of $r$ and $\theta$ respectively. Substituting for and dividing through by $V$, Eq.(\ref{eq:2}) can be written as 
\begin{equation}
\frac{1}{R}\frac{d}{dr}\bigg(r^2 \frac{dR}{dr}\bigg) + \frac{1}{\Theta \sin \theta}\frac{d}{d \theta}\bigg(\sin \theta \frac{d \Theta }{d \theta}\bigg) = 0~.
\label{eq:3}
\end{equation}This form shows that the first term depends only on $r$, and the second only on $\theta$. It follows that  each term is equal to a constant defined, for convenience, to be $s(s+1)$:
\begin{subequations}
\begin{align}
&\frac{d}{dr}\bigg(r^2 \frac{dR}{dr}\bigg) = s (s+1) R~, \label{eq:4.1}\\ 
&\frac{d}{d \theta}\bigg(\sin \theta \frac{d \Theta}{d \theta}\bigg) = - s (s+1) \Theta \sin \theta~. \label{eq:4.2}
\end{align}
\label{eq:4}
\end{subequations} 

The partial differential equation, Eq.(\ref{eq:2}), has been converted into two independent ordinary differential equations. The general solution for the radial equation, Eq.(\ref{eq:4.1}), is:
\begin{equation}
R = Ar^5 + \frac{B}{r^{s+1}},
\end{equation} where $A$ and $B$ are arbitrary constants \cite{RHB-MathematicalMethods}. However, the general solution for the angular equation, Eq.(\ref{eq:4.2}), requires Legendre Polynomials in the variable $\cos \theta$. These are special functions usually encountered in physical problems involving Laplace's equation \cite{RHB-MathematicalMethods}. The solution has the form
\begin{equation}
\Theta(\theta) = P_{s}(\cos \theta)~,
\label{eq:5}
\end{equation} where $P_s$ is the $s$th-order polynomial in $x$ and can be expressed using the Rodrigues' formula \cite{RHB-MathematicalMethods}
\begin{equation}
P_s(x) = \frac{1}{2^s s!} \frac{d^s}{dx^s}(x^2 -1)^s~.
\label{eq:6}
\end{equation}

\noindent Therefore, the most general separable solution is:
\begin{equation}
V(r, \theta) = \bigg(A r^s + \frac{B}{r^{s+1}}\bigg) P_s (\cos \theta)~.
\label{eq:7}
\end{equation}  

\noindent Since Laplace's equation is a linear PDE, the general solution is a superposition of all solutions corresponding to different allowed values of $s$ \cite{griffiths-introElec}. The linear combination can be written as:
\begin{equation}
V(r, \theta) = \sum_{s=0}^{\infty} \bigg( A_s r^s + \frac{B_s}{r^{s+1}}\bigg) P_s (\cos \theta)~,
\label{eq:8}
\end{equation} where $A_s$ and $B_s$ are arbitrary constants that are determined by boundary conditions. \\ \par 

\subsection*{Numerical Techniques}
Finite difference scheme \\
relaxation methods \\
adaptive meshing


\section*{Solving a physical system}

SYSTEM A
Consider system $A$, which has a perfectly uniform electric field defined by two plates at potential $+V$ and $-V$. When a long conducting cylinder is placed into the field at ground potential, the electric field is altered as shown in Fig.(). This system can be described mathematically using principles of electrostatics.

The system $A$ illustrated above can be solved analytically using Laplace's Equation, Eq.(\ref{eq:intro3}). This is a partial differential equation (PDE) that describes the electrostatic potential for a given charge distribution, and consequently the corresponding field. A mathematical method, namely Separation of Variables, can be applied, resulting in a set of ordinary differential equations. Then, using the required boundary conditions, a particular solution describing the diagram in Fig.() may be obtained. \\ \par 

DERIVATION
The uncharged cylinder in Fig.() is a perfect conductor placed in a uniform electric field. The field pushes free positive charges to the right and the negative ones to the left. The charges accumulate in the edges of the conductor distorting the field in the surroundings of the cylinder. The electrostatic potential outside the conductor can be found using the general solution Eq.(\ref{eq:8}) and the required boundary conditions.  \par
The potential at the surface of a grounded conductor is zero \cite{griffiths-introElec}. Also, far from the cylinder the field is perpendicular to the plates, so at infinity $\bf{E}$ = $E_0$$\bf{\hat{x}}$, hence $V = -E_0 x + C$, where C is an arbitrary constant. Due to the symmetry of the system, $V$ is known to be zero at the same distance from the plates. In polar coordinates, this becomes $V(r,\theta) = -E_0 r \cos \theta$. Therefore, the boundary conditions can be expressed as
\begin{subequations}
\begin{align}
&V = 0  \hspace{75pt} r = R~,\\ 
&V \to -E_0 r \cos \theta \hspace{27pt} r \gg R~,
\end{align}
\end{subequations}where $R$ is the radius of the sphere. 

The first boundary condition implies that the general solution is zero, but $P_s(\cos \theta)$ is not always zero, therefore
\begin{equation}
A_s R^s + \frac{B_s}{R^{s+1}} = 0~,
\end{equation}
or
\begin{equation}
B_s = -A_s R^{2s+1}~.
\end{equation}

The general solution, Eq.(\ref{eq:8}) can be rewritten as
\begin{equation}
V(r,\theta) = \sum_{s=0}^{\infty} A_s \bigg(r^s - \frac{R^{2s+1}}{r^{s+1}}\bigg) P_s (\cos \theta)~.
\end{equation}

For $r\gg R$ the term $(\sfrac{R^{2s+1}}{r^{s+1}})$ goes to zero, therefore the second condition requires that
\begin{equation}
V(r,\theta)=\sum_{s=0}^{\infty} A_s r^s P_s(\cos \theta) = -E_0 r \cos \theta~,
\label{eq:14}
\end{equation}since there is only one term in this summation, $s=1$. So $P_s(\cos \theta)= \cos \theta$ and $A_1 = -E_0$ and $A_{s \ne 1} = 0$. Therefore, Eq.(\ref{eq:14}) becomes
\begin{equation}
V(r,\theta) = -E_0 \bigg(r - \frac{R^3}{r^2}\bigg) \cos \theta~.
\end{equation}
The particular solution describing system A is then,
\begin{equation}
V(r,\theta) = \left\{ 
  \begin{array}{l l}
   -E_0 \big(r - \frac{R^3}{r^2}\big) \cos \theta  & \qquad \text{for $r \geq R$}\\
    0 & \qquad \text{for $r < R$}
  \end{array} \right.
\end{equation}


\section*{References}
\bibliographystyle{ieeetr}
\bibliography{Bibliography3rdYear}

\end{document}