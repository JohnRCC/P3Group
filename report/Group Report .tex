\documentclass[a4paper]{jpconf}
\usepackage{amsmath} % for subequations
%\usepackage{graphicx} %for figures
%\usepackage{caption} %for figures



\begin{document}

\title{The effect of solid objects on the electrostatic potential}
\author{J Cork, T  Hendrick-Beattie, D Lafferty, M Pereira, V Nordgren and N Warrack}
\address{School of Physics and Astronomy, University of Glasgow, Glasgow, UK}

\begin{abstract}
\end{abstract}

\section*{Introduction}
Electromagnetism is one of four fundamental forces of nature. It describes how electrically charged particles interact with each other, and how they generate electromagnetic fields \cite{Sears.Zamansky-uniPhy}. These fields permeate the space around them, influencing the behaviour of other charges by exerting a force on them. %(that is either attractive or repulsive). 
Electromagnetic forces dominate most of the physical phenomena encountered in daily life such as sound, light and biological processes. Furthermore, they determines the atomic and macroscopic properties of matter. Understanding these principles at a deep level is not only essential for any physicist but also enables significant advances in science and technology. By successfully modelling a simple configuration of conductors it is possible to investigate arbitrarily complex systems of charges which can be used to design detectors for particle physics. 


%EXTRA
%The electromagnetic force plays a major role in determining the internal properties of most objects encountered in daily life. Ordinary matter takes its form as a result of intermolecular forces between individual molecules in matter. Electrons are bound by electromagnetic wave mechanics into orbitals around atomic nuclei to form atoms, which are the building blocks of molecules. This governs the processes involved in chemistry, which arise from interactions between the electrons of neighboring atoms, which are in turn determined by the interaction between electromagnetic force and the momentum of the electrons.

Electrostatics is the study of stationary or slow-moving electric charges \cite{griffiths-introElec}. They exert electrostatic forces on each other, which in turn are governed by Coulomb's Law and most conveniently described by electric field equations. The relationship between these fields and the distribution of electric charge can be expressed as \cite{griffiths-introElec}
\begin{equation}
\textbf{$\nabla$} \cdot \textbf{E} = \frac{\rho}{\epsilon_0}~,
%\oint \textbf{E} \cdot d\textbf{A} = \frac{Q_{enc}}{\epsilon_0}
\label{eq:intro1}
\end{equation} known as the differential form of Guass's Law, where \textbf{$\nabla$} $\cdot$ \textbf{E} is the divergence of the electric field, $\epsilon_0$ is the permittivity of free space and $\rho$ is the charge density. Additionally, for any static charge distribution the electric field is irrotational, \textbf{$\nabla$} $\times$ \textbf{E} $= 0$. Therefore, the line integral of the electric field, $\oint \textbf{E} \cdot d \textbf{l} = 0$, is independent of the path taken \cite{griffiths-introElec}. This means that \textbf{E} can be written as the gradient of a scalar potential 
\begin{equation}
\textbf{E} = - \nabla \Phi~,
\label{eq:intro2}
\end{equation} where $\Phi$ is called the electric potential, defined as the potential energy per unit charge  \cite{Sears.Zamansky-uniPhy}.
In regions where there is no charge, such that $\rho = 0$, the divergence of the electric field is zero. Hence, using Eq.(\ref{eq:intro1}) and Eq.(\ref{eq:intro2}) the electric potential in all space can be described as
\begin{equation}
\nabla^2 \Phi = 0~.
\label{eq:intro3}
\end{equation} This is Laplace's Equation \cite{RHB-MathematicalMethods}.\\ \par
In this work, the electrostatic potential and the electric field are evaluated for all points in two systems. The systems consist of different geometric configurations of conductors held at constant potentials. The study provides evidence of how the potentials are modified in the presence these conductors. Analytical and numerical techniques, such as the Finite Difference Method, are used to find solutions and their differences are quantified. WHAT WAS ACHIEVED? \\ \par


Consider system $A$, which has a perfectly uniform electric field defined by two plates at potential $+V$ and $-V$. When a long conducting cylinder is placed into the field at ground potential, the electric field is altered as shown in Fig.(). This system can be described mathematically using principles of electrostatics.


\section*{Analytical Approach}
%Partial differential equations (PDE) are fundamental to describe physical phenomena, such as heat, fluids, electrostatics, sound and so on. This is an equation that involves an unknown function of two or more variables and their partial derivatives. A particular solution may be specified from the general solution of a PDE in the presence of boundary conditions. \par
Consider system $A$, which has a perfectly uniform electric field defined by two plates at potential $+V$ and $-V$. When a long conducting cylinder is placed into the field at ground potential, the electric field is altered as shown in Fig.(). This system can be described mathematically using principles of electrostatics.

The system $A$ illustrated above can be solved analytically using Laplace's Equation, Eq.(\ref{eq:intro3}). This is a partial differential equation (PDE) that describes the electrostatic potential for a given charge distribution, and consequently the corresponding field. A mathematical method, namely Separation of Variables, can be applied, resulting in a set of ordinary differential equations. Then, using the required boundary conditions, a particular solution describing the diagram in Fig.() may be obtained. \\ \par 
The electric field is dominated by inherent symmetries that are better described in spherical coordinates \cite{RHB-MathematicalMethods}, in which Laplace's Equation is expressed as

\begin{equation}
\frac{1}{r^2}\frac{\partial}{\partial r}\bigg(r^2 \frac{\partial V}{\partial r}\bigg) + \frac{1}{r^2 \sin \theta} \frac{\partial}{\partial \theta}\bigg(\sin \theta \frac{\partial V}{\partial \theta}\bigg) + \frac{1}{r^2 \sin^2 \theta}\frac{\partial^2 V}{\partial \phi^2} = 0~.
\label{eq:1}
\end{equation}

By noting that $V$ is independent of $\phi$ due to azimuthal symmetry, and by multiplying every term by $r^2$ the equation can be simplified to
\begin{equation}
\frac{\partial}{\partial r}\bigg(r^2 \frac{\partial V}{\partial r}\bigg) + \frac{1}{\sin \theta}\frac{\partial}{\partial \theta}\bigg(\sin \theta \frac{\partial V}{\partial \theta}\bigg) = 0~.
\label{eq:2}
\end{equation}

Separation of Variables is used to solve the equation above. Assume that $V(r,\theta) = R(r)\Theta(\theta)$, where the factors $R$ and $\Theta$ are functions of $r$ and $\theta$ respectively. Then, by substituting and dividing by $V$, Eq.(\ref{eq:2}) can be written as 
\begin{equation}
\frac{1}{R}\frac{d}{dr}\bigg(r^2 \frac{dR}{dr}\bigg) + \frac{1}{\Theta \sin \theta}\frac{d}{d \theta}\bigg(\sin \theta \frac{d \Theta }{d \theta}\bigg) = 0~.
\label{eq:3}
\end{equation}

This form shows that the first term only depends on $r$, and the second only on $\theta$. It follows that  each term is equal to a constant defined for convenience to be $s(s+1)$, hence
\begin{subequations}
\begin{align}
&\frac{d}{dr}\bigg(r^2 \frac{dR}{dr}\bigg) = s (s+1) R~, \label{eq:4.1}\\ 
&\frac{d}{d \theta}\bigg(\sin \theta \frac{d \Theta}{d \theta}\bigg) = - s (s+1) \Theta \sin \theta~. \label{eq:4.2}
\end{align}
\label{eq:4}
\end{subequations} 

The partial differential equation, Eq.(\ref{eq:2}), has been converted into two independent ordinary differential equations. The general solution for the radial equation, Eq.(\ref{eq:4.1}), is \cite{RHB-MathematicalMethods}:
\begin{equation}
R = Ar^5 + \frac{B}{r^{s+1}},
\end{equation} where $A$ and $B$ are arbitrary constants. However, the general solution for the angular equation, Eq.(\ref{eq:4.2}), requires Legendre Polynomials in the variable $\cos \theta$. These are special functions usually encountered in physical problems involving Laplace's Equation \cite{RHB-MathematicalMethods}. The solution has the form
\begin{equation}
\Theta(\theta) = P_{s}(\cos \theta)~,
\label{eq:5}
\end{equation} where $P_s$ is the $s$th-order polynomial in $x$ and can be expressed using the Rodrigues' formula \cite{RHB-MathematicalMethods}
\begin{equation}
P_s(x) = \frac{1}{2^s s!} \frac{d^s}{dx^s}(x^2 -1)^s~.
\end{equation}

Therefore, the most general separable solution is:
\begin{equation}
V(r, \theta) = \bigg(A r^s + \frac{B}{r^{s+1}}\bigg) P_s (\cos \theta)~.
\end{equation}  

Since Laplace's Equation is a linear PDE, the general solution is a superposition of all solutions corresponding to different allowed values of $s$ \cite{griffiths-introElec}. The linear combination can be written as:
\begin{equation}
V(r, \theta) = \sum_{s=0}^{\infty} \bigg( A_s r^s + \frac{B_s}{r^{s+1}}\bigg) P_s (\cos \theta)~,
\end{equation} where $A_s$ and $B_s$ are arbitrary constants. \\ \par 

A particular solution can be obtained using boundary conditions, namely
\begin{subequations}
\begin{align}
&V = 0  \hspace{75pt} r = R~,\\ 
&V \to -E_0 r \cos \theta \hspace{27pt} r \gg R~.
\end{align}
\end{subequations}

The first boundary condition comes from the potential
\begin{equation}
A_s R^s + \frac{B_s}{R^{s+1}} = 0 
\end{equation}
or
\begin{equation}
B_s = -A_s R^{2s+1}
\end{equation}
so
\begin{equation}
V(r,\theta) = \sum_{s=0}^{\infty} A_s \bigg(r^s - \frac{R^{2s+1}}{r^{s+1}}\bigg) P_s (\cos \theta)
\end{equation}

BC2
\begin{equation}
\sum_{s=0}^{\infty} A_s \bigg(r^s - \frac{R^{2s+1}}{r^{s+1}}\bigg) P_s (\cos \theta)
\end{equation}
becomes
\begin{equation}
\sum_{s=0}^{\infty} A_s r^s P_s(\cos \theta) = -E_0 r \cos \theta
\end{equation}
evidently
\begin{equation}
P_s (\cos \theta) = \cos \theta 
\end{equation}
ie. $s=1$. then, 
\begin{equation}
A_{s=1} = - E_0, \hspace{30pt} A_{s \neq 1} = 0
\end{equation}
therefore
\begin{equation}
V(r,\theta) = -E_0 \bigg(r - \frac{R^3}{r^2}\bigg) \cos \theta
\end{equation}
general solution
\begin{equation}
f(n) = \left\{ 
  \begin{array}{l l}
   V(r,\theta) = -E_0 \bigg(r - \frac{R^3}{r^2}\bigg) \cos \theta  & \qquad \text{for $r \geq R$}\\
    0 & \qquad \text{for $r < R$}
  \end{array} \right.
\end{equation}
\section*{Numerical Techniques}

\section*{References}
\bibliographystyle{ieeetr}
\bibliography{Bibliography3rdYear}

\end{document}